\section{Quantum Dynamics: The Jaynes-Cummings Model}
\label{sec:quantum}

We demonstrate that the $\NA$ correction term in quantum reduced dynamics
aligns with the Nakajima-Zwanzig memory-kernel structure under the stated
projection and initial-state assumptions. The Jaynes-Cummings
model provides an exactly solvable test case.

\subsection{The Jaynes-Cummings Model}

The Jaynes-Cummings (JC) Hamiltonian describes a two-level atom coupled to a
single quantized field mode:
\begin{equation}
  H = \hbar\omega_a \sigma^+\sigma^- + \hbar\omega_f a^\dagger a + \hbar g(\sigma^+ a + \sigma^- a^\dagger)
\end{equation}
where $\sigma^\pm$ are atomic raising/lowering operators, $a^\dagger, a$ are
photon creation/annihilation operators, and $g$ is the coupling strength.

At resonance ($\omega_a = \omega_f$), starting from $|e, n\rangle$ (excited atom,
$n$ photons), the exact evolution is:
\begin{equation}
  |\psi(t)\rangle = \cos(\Omega_n t)|e, n\rangle - i\sin(\Omega_n t)|g, n+1\rangle
\end{equation}
where $\Omega_n = g\sqrt{n+1}$ is the generalized Rabi frequency.

\subsection{Projection: Tracing Out the Field}

The reduced atomic density matrix is obtained by tracing over the field:
\begin{equation}
  \rho_A(t) = \mathrm{Tr}_F[|\psi(t)\rangle\langle\psi(t)|]
\end{equation}

The probability of finding the atom excited is:
\begin{equation}
  P_e^{\text{full}}(t) = \cos^2(\Omega_n t)
\end{equation}

This is Rabi oscillation: the atom periodically exchanges excitation with the field.

\subsection{Naive Projected Dynamics}

If we project first and then evolve, we trace out the field at $t=0$:
\begin{equation}
  \rho_A(0) = |e\rangle\langle e|
\end{equation}
The projected dynamics uses only the atomic Hamiltonian, which (after tracing
out the field) has no remaining interaction. The naive prediction is:
\begin{equation}
  P_e^{\text{naive}}(t) = 1 \quad \text{(constant, incorrect)}
\end{equation}

The naive projected model completely misses the Rabi oscillations.

\subsection{Projection Timing Sensitivity}

This is a textbook example of PTS:
\begin{equation}
  \mathrm{PTS} = P_e^{\text{full}}(t) - P_e^{\text{naive}}(t) = \cos^2(\Omega_n t) - 1 = -\sin^2(\Omega_n t)
\end{equation}

The PTS reaches maximum magnitude $1$ at $t = \pi/(2\Omega_n)$, when the atom
is maximally entangled with the field.

\paragraph{Projection timing sensitivity in the quantum channel.}
We compute a remainder-channel timing signal by comparing the outcomes of
(i) projecting and then evolving versus (ii) evolving and then projecting.
Even in a simple JC setting, the divergence is time-structured rather than
a single scalar; this supports treating projection as a first-class operation
whose timing can be audited.

\begin{figure}[t]
  \centering
  \includegraphics[width=\linewidth]{figures/quantum/plot_02_pts_divergence.png}
  \caption{Quantum PTS example. Timing divergence as a function of projection time,
  illustrating $P \circ U \neq U \circ P$. An admissibility criterion
  based on an entropy threshold produces time windows where totalization is
  permitted vs refused (fail-closed).}
  \label{fig:jc_pts_windows}
\end{figure}

\subsection{The NA0 Correction Term}

The $\NA$ framework identifies the correction needed:
\begin{equation}
  \frac{d P_e}{dt} = -\Omega_n \sin(2\Omega_n t)
\end{equation}

This is the ``memory kernel'' contribution from the traced-out field. Integrating:
\begin{equation}
  P_e^{\text{corrected}}(t) = 1 + \int_0^t \frac{d P_e}{d\tau}\, d\tau = \cos^2(\Omega_n t) = P_e^{\text{full}}(t)
\end{equation}

The $\NA$ correction exactly recovers the full dynamics.

\subsection{NA0 Representation}

The partial trace projection produces:
\begin{equation}
  \proj_{\mathrm{Tr}_F}(|\psi\rangle\langle\psi|) =
  \NA\left\langle \rho_{\text{corr}}(t), \rho_A(t); \mathrm{Tr}_F \right\rangle
\end{equation}
where $\rho_{\text{corr}}$ contains the system-field correlations that were
traced out.

The debt is:
\begin{itemize}
  \item The off-diagonal elements of the full density matrix in the product basis
  \item Equivalently, the entanglement between atom and field
  \item Quantified by the von Neumann entropy: $S(\rho_A) = -\mathrm{Tr}[\rho_A \log \rho_A]$
\end{itemize}

\subsection{Experimental Verification}

\begin{figure}[t]
  \centering
  \includegraphics[width=0.9\textwidth]{figures/quantum/jc_na0_dynamics.png}
  \caption{Jaynes-Cummings NA0 dynamics.
    \textbf{Top}: Excited state probability $P_e(t)$ for full model (blue),
    naive projected model (red), and NA0-corrected model (green, overlaps blue).
    \textbf{Middle}: Purity of reduced atomic state, showing periodic entanglement.
    \textbf{Bottom}: NA0 correction rate, peaking when entanglement is maximum.}
  \label{fig:jc-dynamics}
\end{figure}

\Cref{fig:jc-dynamics} shows numerical results from our JC simulation:
\begin{itemize}
  \item The full and corrected models agree exactly (to numerical precision)
  \item The naive model has $100\%$ error at half-periods
  \item The correction rate $|dP_e/dt|$ correlates with entanglement (purity minimum)
\end{itemize}

\paragraph{Open-system extension: dissipation (Lindblad).}
Real cavities are not closed: photon loss and dephasing suppress coherent exchange.
We include a minimal Lindblad decay model (cavity decay rate $\kappa$) and track the
resulting reduction in visible Rabi oscillations as a function of $\kappa$.
This is a direct stress-test of projection-honest bookkeeping: when coherence is
lost, the reduced description must either (i) carry explicit debt describing the
discarded correlations, or (ii) fail closed rather than silently totalize.

\begin{figure}[t]
  \centering
  \includegraphics[width=\linewidth]{figures/quantum/plot_07_dissipation.png}
  \caption{Dissipation in the Jaynes--Cummings model. Increasing cavity decay
  $\kappa$ suppresses Rabi oscillations in $P_e(t)$. The transition from strong
  to weak coupling is visible near $\kappa \approx g$, beyond which oscillations
  become effectively unobservable.}
  \label{fig:jc_dissipation}
\end{figure}

\subsection{Connection to Nakajima-Zwanzig}

\paragraph{Assumptions (for this section).}
We consider (i) a specified projection superoperator $P$ (e.g., $P\rho = \rho_S \otimes \rho_E^{eq}$),
(ii) an initial product state (system uncorrelated with environment at $t=0$),
and (iii) a chosen observable channel when we present scalar kernels (as distinct from the full superoperator kernel).
Within these assumptions, the NA0 ``debt'' term captures the information discarded by naive projection and yields a correction term with the same structural role as the NZ memory contribution.

The Nakajima-Zwanzig equation~\cite{nakajima1958,zwanzig1960} provides the
exact reduced dynamics:
\begin{equation}
  \frac{d\rho_S}{dt} = \mathcal{L}_S \rho_S + \int_0^t K(t-\tau) \rho_S(\tau)\, d\tau + I(t)
\end{equation}

The terms are:
\begin{itemize}
  \item $\mathcal{L}_S$: Liouvillian from the projected Hamiltonian (naive dynamics)
  \item $K(t)$: Memory kernel encoding back-action from the environment
  \item $I(t)$: Inhomogeneity from initial correlations
\end{itemize}

In $\NA$ terms:
\begin{itemize}
  \item The naive dynamics uses only the first term (totalizing the debt)
  \item The memory kernel is the derivative of accumulated debt
  \item $\NA$-correct dynamics includes all terms
\end{itemize}

In the specific JC setup studied here (resonance, initial product state, $n=0$ Fock component
where $\Omega_0 = g$), the NA0 correction on the $P_e$ observable channel plays the structural
role of a memory kernel. From Eq.~(35), the correction rate is:
\[
  \mathrm{NA0}(t) = \frac{dP_e}{dt} = -g\sin(2gt).
\]
Integrating restores the full Rabi oscillation:
\begin{equation}
  P_e^{\text{full}}(t) = P_e^{\text{naive}} + \int_0^t \mathrm{NA0}(\tau)\, d\tau
  = 1 + \left[\frac{\cos(2gt)}{2} - \frac{1}{2}\right] = \cos^2(gt).
\end{equation}
In the scalar observable channel we track, the NA0 correction plays the same structural role as
the NZ correction term (restoring the reduced dynamics after naive projection), though the full
NZ object is a superoperator-valued convolution kernel. To be explicit: we match the structural
role and reproduce the observable, not derive $K(t)$ for the full reduced density matrix.

\subsection{Scaling Behaviors}

\begin{figure}[t]
  \centering
  \includegraphics[width=0.8\textwidth]{figures/quantum/photon_scaling.png}
  \caption{Photon number scaling. Higher initial photon number $n$ gives faster
    Rabi oscillations ($\Omega_n = g\sqrt{n+1}$) and larger instantaneous NA0
    correction rates.}
  \label{fig:photon-scaling}
\end{figure}

The NA0 correction scales with system parameters:
\begin{itemize}
  \item \textbf{Coupling strength}: $|\mathrm{NA0}| \propto g$
  \item \textbf{Photon number}: $|\mathrm{NA0}| \propto \sqrt{n+1}$
  \item \textbf{Cumulative error}: $\int |P_e^{\text{naive}} - P_e^{\text{full}}| dt \propto t$
\end{itemize}

\Cref{fig:photon-scaling} shows this scaling. The naive model becomes increasingly
wrong as either coupling strength or photon number increases.

\subsection{Collapse and Revival}

For coherent-state initial conditions $|\alpha\rangle$ with mean photon number
$\bar{n} = |\alpha|^2$, the dynamics shows collapse and revival:
\begin{itemize}
  \item \textbf{Collapse} ($t \sim 1/g$): Dephasing of Fock components causes
    $P_e \to 0.5$
  \item \textbf{Revival} ($t \sim 2\pi\sqrt{\bar{n}}/g$): Partial rephasing
    restores oscillations
\end{itemize}

During collapse, the naive model has persistent $\sim 50\%$ error. The $\NA$
correction tracks this exactly.

\subsection{Implications}

The JC example demonstrates that:
\begin{enumerate}
  \item Projection creates debt (entanglement information)
  \item Debt has back-action on the reduced dynamics (memory kernel)
  \item Ignoring debt gives qualitatively wrong predictions (no Rabi oscillations)
  \item $\NA$-aware dynamics recovers the exact answer
\end{enumerate}

This is not specific to JC. Any open quantum system exhibits similar behavior:
tracing out environmental degrees of freedom creates debt that manifests as
non-Markovian corrections to the reduced dynamics.
