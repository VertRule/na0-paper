\section{Discussion}
\label{sec:discussion}

We have introduced the $\NA$ (Non-Atomic Zero) framework and demonstrated its
application across four domains. Here we discuss implications, limitations,
and future directions.

\subsection{Summary of Results}

The $\NA$ framework provides a unified treatment of projection-induced artifacts:

\paragraph{Divergent Series.} Regularization is a projection policy; the
``value'' $-1/12$ of $\sum n$ is policy-dependent remainder with implicit
divergent debt.

\paragraph{Signal Processing.} Baseline subtraction policies induce systematic,
correlated debt-remainder variations that can masquerade as physical disagreements.

\paragraph{Spectral Analysis.} Eigenvector export carries sign and basis
ambiguity as implicit debt; projector-based export makes this explicit.

\paragraph{Quantum Dynamics.} The Nakajima-Zwanzig memory kernel is the
$\NA$ correction term for traced-out environmental degrees of freedom.

\subsection{Projection-Honest Computation}

We propose \emph{projection-honest computation} as a design principle:
\begin{quote}
  Every projection operation should produce an $\NA$ object, making explicit
  what was discarded and under what policy.
\end{quote}

This enables:
\begin{itemize}
  \item \textbf{Audit trails}: Track how debt accumulated through a computation
  \item \textbf{Policy comparison}: Detect when results depend on projection choices
  \item \textbf{Fail-closed safety}: Refuse to totalize when debt is too high
  \item \textbf{Reproducibility}: Document policies alongside results
\end{itemize}

\subsection{Relation to Existing Work}

The $\NA$ framework connects to several existing concepts:

\paragraph{Information Theory.} The data processing inequality states that
projections cannot increase information. $\NA$ makes the information loss
(debt) explicit rather than implicit.

\paragraph{Numerical Analysis.} Condition numbers quantify sensitivity to
input perturbations. $\NA$ extends this to sensitivity to projection policy,
a different axis of instability.

\paragraph{Open Quantum Systems.} The Nakajima-Zwanzig formalism has been
used for decades. $\NA$ generalizes its insight---that projection creates
correctable debt---beyond quantum mechanics.

\paragraph{Regularization Theory.} Hadamard finite parts and zeta-regularization
are well-established. $\NA$ provides a common framework and notation.

\subsection{Limitations}

\paragraph{Debt Representation.} The $\NA$ framework requires choosing how to
represent debt. For some projections (e.g., partial traces in infinite dimensions),
the debt may be unwieldy or ill-defined. Practical applications require
domain-specific debt formats.

\paragraph{Composition Complexity.} As projections compose, debt accumulates.
For long pipelines, debt management may become computationally expensive.
Approximations or periodic debt truncation may be needed.

\paragraph{Policy-Free Projection.} In some cases, there is no meaningful
alternative policy---the projection is canonical. $\NA$ is most useful when
multiple reasonable policies exist.

\paragraph{Novelty Scope.} Much of what $\NA$ captures is ``known'' in specific
domains (regularization, open systems, numerical conditioning). The contribution
is unification and explicit representation, not new physics or mathematics.

\paragraph{Driven systems and boundary cases.}
NA0 as presented assumes a declared projection policy with a stable remainder/lift
interface over the relevant time horizon. In strongly driven regimes (time-dependent
Hamiltonians) and nonlinear measurement backaction scenarios, qualitative phenomena
such as photon blockade can emerge in ways that are not captured by a fixed
projection map without extending the policy to be time-indexed or augmenting the
state with additional tracked variables. In these regimes, we recommend explicit
fail-closed thresholds rather than silent totalization.

\subsection{Future Work}

\paragraph{Software Implementation.} A reference implementation of $\NA$ objects
with debt tracking, composition, and fail-closed policies would enable adoption.
Integration with numerical libraries (NumPy, SciPy) is a natural next step.

\paragraph{Domain Applications.} The framework should be tested in specific
high-stakes domains:
\begin{itemize}
  \item Cosmology: Does the Hubble tension have a projection-policy component?
  \item Machine learning: Do hyperparameter choices function as projection policies?
  \item Financial modeling: How does aggregation policy affect risk estimates?
\end{itemize}

\paragraph{Formal Verification.} Can we prove that a computation is
projection-honest? Type systems or proof assistants might enforce debt tracking.

\paragraph{Optimal Policy Selection.} Given a task, can we identify the
``best'' projection policy? This requires formalizing what makes a policy good
(minimum debt? Minimum variance? Maximum interpretability?).

\subsection{Implications for Scientific Practice}

The $\NA$ framework suggests practical changes to scientific workflow:

\begin{enumerate}
  \item \textbf{Report policies, not just results.} Papers should document
    projection choices (baseline model, regularization scheme, eigenvector
    convention) alongside numerical results.

  \item \textbf{Test policy sensitivity.} Before claiming a result is robust,
    vary projection policies and check for debt-remainder correlation.

  \item \textbf{Distinguish tension types.} ``Tension'' between experiments
    should be classified: statistical (noise), systematic (calibration),
    physical (new phenomena), or projection-induced (policy disagreement).

  \item \textbf{Use fail-closed defaults.} When debt is high or ambiguous,
    refuse to report a scalar. Instead, report the debt structure explicitly.
\end{enumerate}

\subsection{Conclusion}

Projection is ubiquitous, and the information it discards does not disappear.
The $\NA$ framework makes this debt explicit, enabling detection and management
of projection-induced artifacts. We hope this contributes to more reproducible
and auditable scientific computation.

\subsection{Reproducibility}

All experiments in this paper are deterministically reproducible. The artifact bundle includes:

\paragraph{Environment.} Python 3.9+, NumPy, SciPy, Matplotlib. Dependencies are pinned in \texttt{requirements.txt}. No external randomness is used; all ``random'' initialization uses fixed seeds.

\paragraph{Execution.} Each experiment can be reproduced via:
\begin{itemize}
  \item \textbf{Signal processing}: \texttt{python projection\_experiment.py}
  \item \textbf{Spectral analysis}: \texttt{python spectral\_na0.py}
  \item \textbf{Quantum dynamics}: \texttt{python na0/jc-na0-demo/jc\_plots.py}
\end{itemize}

\paragraph{Determinism guarantees.}
Output files use sorted column order in CSV exports, fixed-precision float formatting (6 decimal places), and stable dictionary iteration (Python 3.7+ insertion order). Figure generation uses fixed random seeds where applicable.

\paragraph{Verification.}
The benchmark harness (\texttt{make bench} or \texttt{python na0\_benchmark.py}) runs
four reproducibility tests with pass/fail status and SHA-256 output hashes:

\begin{center}
\small
\begin{tabular}{@{}llcc@{}}
\toprule
\textbf{Benchmark} & \textbf{Test} & \textbf{Threshold} & \textbf{Determinism} \\
\midrule
\texttt{spectral} & Projector vs eigenvector variance ratio & $< 0.1$ & \checkmark \\
\texttt{signal} & Debt-variance correlation & $> 0.7$ & \checkmark \\
\texttt{quantum} & PTS max timing accuracy & $< 5\%$ & \checkmark \\
\texttt{determinism} & Cross-run hash equality & exact & \checkmark \\
\bottomrule
\end{tabular}
\end{center}

All benchmarks produce identical SHA-256 hashes across runs with identical inputs.
The complete artifact bundle, including source code, benchmark harness, and generated figures, will be released with a stable URL.

\paragraph{Collaboration.}
The author welcomes collaboration on projection-honest computation, numerical
stability, and related invariant-driven programs; correspondence is invited at
\texttt{Dave@vertrule.com}.
