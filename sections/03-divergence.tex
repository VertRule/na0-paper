\section{Divergence and Regularization}
\label{sec:divergence}

The $\NA$ framework provides a natural language for divergent series
regularization, clarifying statements like ``$1 + 2 + 3 + \cdots = -1/12$''.

\subsection{The Problem of Divergent Sums}

The series $\sum_{n=1}^\infty n = 1 + 2 + 3 + \cdots$ diverges. Yet in
physics---particularly string theory, Casimir effect calculations, and
zeta-function regularization---this series is assigned the value $-1/12$.
This is not a claim about convergence but about \emph{regularization}:
extracting a finite part from a divergent object.

\subsection{Regulated Families and NA0}
\label{sec:div_regulated}

To keep debt as a mathematically well-typed object, we represent divergent expressions via a regulated family $S(\epsilon)$ and a declared scheme/policy $\Pi$.
The remainder is the scheme-defined finite part, and the debt is the discarded divergent asymptotic data together with the regulator identity.

\paragraph{Example: exponential regulator.}
Consider the regulated series
\[
  S(\epsilon) \coloneqq \sum_{n=1}^\infty n e^{-\epsilon n}, \quad \epsilon>0.
\]
This admits a closed form $S(\epsilon)=\frac{e^{-\epsilon}}{(1-e^{-\epsilon})^2}$ and has the asymptotic expansion as $\epsilon\to 0^+$:
\[
  S(\epsilon) = \frac{1}{\epsilon^2} - \frac{1}{12} + O(\epsilon^2).
\]
Under a ``finite-part'' policy $\Pi_{\mathrm{FP}}$ that retains the constant term and records the divergent terms as debt, we define:
\[
  P_{\Pi_{\mathrm{FP}}}(S) \;=\;
  \NA\Big\langle D(\epsilon),\; R;\; \Pi_{\mathrm{FP}}\Big\rangle,
\]
where the remainder is the finite part $R=-\frac{1}{12}$ and the debt is the discarded divergent asymptotic series plus regulator metadata, e.g.
\[
  D(\epsilon) = \left(\frac{1}{\epsilon^2} + O(\epsilon^2),\; \text{regulator}=\text{exp},\; \text{retained term}=\epsilon^0\right).
\]
This makes explicit what is usually implicit: the value $-\frac{1}{12}$ is a policy-dependent remainder extracted from a regulated family, and the divergent structure is retained as a first-class object rather than silently discarded.

\subsection{Zeta-Function Regularization}

The Riemann zeta function provides a canonical regularization via analytic continuation.
For $\Re(s) > 1$:
\begin{equation}
  \zeta(s) = \sum_{n=1}^\infty \frac{1}{n^s}
\end{equation}
extended to the entire complex plane (except $s=1$) by analytic continuation.
At $s = -1$, $\zeta(-1) = -\frac{1}{12}$.

In NA0 terms, zeta-regularization is a policy $\Pi_\zeta$ where:
\begin{itemize}
  \item The ambient space $\mathcal{X}$ is regulated families (formal power series in a regulator parameter)
  \item The remainder extractor $r_{\Pi_\zeta}$ returns the analytically-continued value
  \item The debt includes the divergent asymptotic terms and the choice of analytic continuation path
\end{itemize}

\subsection{Alternative Regularizations}

Different policies yield different remainders:

\paragraph{Ramanujan Summation.} Ramanujan's method assigns
$\sum_{n=1}^\infty n = -1/12$ via a specific definition involving the
Euler-Maclaurin formula. The debt structure differs from zeta-regularization
even though the remainder is the same.

\paragraph{Cutoff Regularization.} Introducing a cutoff $N$ and taking
$N \to \infty$:
\begin{equation}
  \sum_{n=1}^N n = \frac{N(N+1)}{2} \sim \frac{N^2}{2} + \frac{N}{2}
\end{equation}
This diverges; extracting a finite part requires subtracting the divergent
terms, which involves arbitrary choices. Different subtraction schemes give
different remainders.

\paragraph{Dimensional Regularization.} In $d$ dimensions, certain sums become
convergent and can be analytically continued. The remainder depends on the
dimension.

\subsection{Debt-Remainder Correlation}

A key $\NA$ prediction: changing the regularization policy should produce
correlated changes in debt and remainder. If a policy change shifts the
remainder by $\Delta\remainder$, it must shift the debt by $-\Delta\remainder$
(for additive projections) to conserve information.

This can be tested: vary the regularization policy (e.g., cutoff scale,
analytic continuation path) and measure how debt and remainder co-vary.

\begin{figure}[t]
  \centering
  \includegraphics[width=\textwidth]{figures/divergence/regulator_validation.png}
  \caption{Numerical validation of debt-remainder separation in zeta-regularization.
    \textbf{Left}: Regulated sum $S(\varepsilon)$ and divergent term $1/\varepsilon^2$.
    \textbf{Center}: Residual after subtracting the divergent term; the extracted constant
    converges to $-1/12$ with relative error $< 0.001\%$.
    \textbf{Right}: Debt magnitude (divergent term) vs remainder (finite part), showing
    the stable remainder despite varying debt. Script: \texttt{divergence\_validation.py}.}
  \label{fig:regulator-validation}
\end{figure}

\Cref{fig:regulator-validation} provides numerical validation: we compute
$S(\varepsilon) = \sum_{n=1}^\infty n e^{-\varepsilon n}$ for decreasing $\varepsilon$,
subtract the divergent $1/\varepsilon^2$ term, and extract the constant.
The extracted remainder converges to $-1/12$ with relative error $< 0.001\%$,
demonstrating that the debt-remainder decomposition is numerically stable
and policy-dependent (the constant changes if we change which asymptotic terms
are retained as debt).

\subsection{Famous Formulas}

Several ``famous'' regularized values can be stated precisely in $\NA$ form:

\begin{align}
  \proj_\zeta(1 + 1 + 1 + \cdots) &= \NA\langle \cdot, -\tfrac{1}{2}; \zeta \rangle \\
  \proj_\zeta(1 + 2 + 3 + \cdots) &= \NA\langle \cdot, -\tfrac{1}{12}; \zeta \rangle \\
  \proj_\zeta(1 + 4 + 9 + \cdots) &= \NA\langle \cdot, 0; \zeta \rangle \\
  \proj_\zeta(1^3 + 2^3 + 3^3 + \cdots) &= \NA\langle \cdot, \tfrac{1}{120}; \zeta \rangle
\end{align}

where $\langle \cdot, \ldots \rangle$ indicates that the debt is ``everything
else'' required to make the decomposition exact.

\subsection{Physical Applications}

In physics, regularized sums appear in:

\paragraph{Casimir Effect.} The vacuum energy between conducting plates involves
$\sum_{n=1}^\infty n$, regularized to give a finite, measurable force.

\paragraph{String Theory.} The critical dimension $d=26$ for bosonic strings
emerges from requiring $\zeta(-1) = -1/12$ in a consistency calculation.

\paragraph{Quantum Field Theory.} Divergent loop integrals are regularized via
dimensional regularization, yielding finite remainders that match experiment.

In each case, the physics is encoded not in the bare divergent sum but in the
\emph{policy-dependent remainder}. The $\NA$ framework makes this dependence
explicit, rather than treating regularization as a ``trick'' that magically
produces answers.

\subsection{The Hadamard Finite Part}

For divergent integrals, the Hadamard finite part provides a canonical
regularization. For $\int_0^1 x^{-\alpha} dx$ with $\alpha > 1$:
\begin{equation}
  \mathrm{Pf}\int_0^1 x^{-\alpha}\, dx = \frac{1}{1-\alpha}
\end{equation}
interpreted as the analytic continuation from $\alpha < 1$.

In $\NA$ terms, regularizing by cutoff $\epsilon$:
\[
  \int_\epsilon^1 x^{-\alpha}\, dx = \frac{1 - \epsilon^{1-\alpha}}{1-\alpha}
  = \underbrace{\frac{1}{1-\alpha}}_{\text{finite part}}
  + \underbrace{\left(-\frac{\epsilon^{1-\alpha}}{1-\alpha}\right)}_{\text{divergent as }\epsilon\to 0}.
\]
Thus:
\begin{equation}
  \proj_{\mathrm{Hadamard}}\left(\int_0^1 x^{-\alpha}\, dx\right) =
  \NA\left\langle -\frac{\epsilon^{1-\alpha}}{1-\alpha},\; \frac{1}{1-\alpha};\; \mathrm{Hadamard}\right\rangle
\end{equation}
where the debt is the divergent asymptotic term (plus regulator metadata: cutoff at $\epsilon$).
