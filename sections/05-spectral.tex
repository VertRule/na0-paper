\section{Spectral Analysis: Projection-Honest Eigenvectors}
\label{sec:spectral}

Eigendecomposition is a fundamental projection operation in data analysis
(PCA, spectral clustering, etc.). We show that standard eigenvector export
introduces artifacts that $\NA$-aware methods avoid.

\subsection{The Eigenvector Export Problem}

Given a symmetric matrix $A \in \mathbb{R}^{n \times n}$, the eigendecomposition
is:
\begin{equation}
  A = V \Lambda V^T = \sum_{i=1}^n \lambda_i v_i v_i^T
\end{equation}
Standard practice exports the eigenvectors $\{v_i\}$ and eigenvalues $\{\lambda_i\}$.
This creates several artifacts.

\paragraph{Sign Ambiguity.} If $v$ is an eigenvector, so is $-v$. Different
algorithms, initializations, or numerical precision can flip signs arbitrarily.
Comparing eigenvectors across runs or methods produces ``false differences''
that are artifacts, not signal.

\paragraph{Degeneracy Rotation.} When eigenvalues are degenerate or
near-degenerate ($\lambda_i \approx \lambda_j$), the corresponding eigenvectors
span a subspace but are not individually determined. Any rotation within the
subspace is equally valid. Small perturbations can cause large rotations.

\paragraph{Ordering Artifacts.} Eigenvalue ordering conventions (descending,
ascending, by absolute value) are arbitrary. Tracking eigenvectors across
time or conditions requires matching, which can fail near crossings.

\subsection{Projection-Honest Alternative: Projectors}

The fundamental object in spectral analysis is not the eigenvector but the
\emph{projector}:
\begin{equation}
  P_i = v_i v_i^T
\end{equation}
The projector is invariant under sign flips: $(-v)(-v)^T = v v^T$.

For near-degenerate eigenvalues, we should export the \emph{cluster projector}:
\begin{equation}
  P_{\text{cluster}} = \sum_{i \in \text{cluster}} v_i v_i^T
\end{equation}
which projects onto the full eigenspace rather than choosing an arbitrary
basis within it.

\subsection{NA0 Representation}

Standard eigenvector export is a projection with implicit debt:
\begin{equation}
  \proj_{\text{evec}}(A) = \NA\langle \text{sign choice}, \{(\lambda_i, v_i)\}; \text{eigenvector}\rangle
\end{equation}

Projector-based export makes the debt explicit:
\begin{equation}
  \proj_{\text{proj}}(A) = \NA\langle \text{basis choice within eigenspace}, \{(\lambda_i, P_i)\}; \text{projector}\rangle
\end{equation}

For clustered eigenvalues, the fail-closed policy refuses to export individual
eigenvectors:
\begin{equation}
  \proj_{\text{fail-closed}}(A) =
  \begin{cases}
    \{(\lambda_i, v_i)\} & \text{if } |\lambda_i - \lambda_j| > \theta\ \forall i \neq j \\
    \text{cluster projector} & \text{otherwise}
  \end{cases}
\end{equation}

\subsection{Experiments}

We demonstrate four spectral projection artifacts and their $\NA$ solutions.

\subsubsection{Sign-Flip Artifact}

\begin{figure}[t]
  \centering
  \includegraphics[width=0.8\textwidth]{figures/spectral/sign_flip.png}
  \caption{Sign-flip artifact in PCA. Running PCA on identical data with
    different random seeds produces eigenvectors with flipped signs (left).
    Exporting projectors $P = vv^T$ instead (right) yields identical results.}
  \label{fig:sign-flip}
\end{figure}

\Cref{fig:sign-flip} shows PCA on identical data with different random seeds.
Eigenvectors appear different (sign flips) while projectors are identical.
Standard comparison metrics (correlation, angle) would report ``differences''
that are pure artifacts.

\subsubsection{Degeneracy Rotation}

\begin{figure}[t]
  \centering
  \includegraphics[width=0.8\textwidth]{figures/spectral/degeneracy_rotation.png}
  \caption{Degeneracy-induced rotation. A matrix with near-degenerate eigenvalues
    (gap $\epsilon = 10^{-4}$) is perturbed by $10^{-6}$. Eigenvectors rotate
    dramatically (left); the span (cluster projector) is stable (right).}
  \label{fig:degeneracy}
\end{figure}

\Cref{fig:degeneracy} demonstrates that tiny perturbations cause large eigenvector
rotations when eigenvalues are close. The 2D eigenspace (cluster projector) is
stable; only the basis within it changes.

\subsubsection{Procrustes Tracking}

\begin{figure}[t]
  \centering
  \includegraphics[width=0.9\textwidth]{figures/spectral/procrustes_tracking.png}
  \caption{Time series eigenvector tracking. Raw eigenvectors (top) show
    discontinuities from sign flips and mode crossings. Procrustes alignment
    (middle) produces smooth evolution. Subspace distance $d(P_t, P_{t+1})$
    (bottom) provides a policy-independent stability metric.}
  \label{fig:procrustes}
\end{figure}

For time series of covariance matrices, tracking eigenvectors requires a
\emph{policy}: how to match eigenvectors across time points. \Cref{fig:procrustes}
compares:
\begin{itemize}
  \item \textbf{Raw export}: Discontinuities from sign flips and mode crossings
  \item \textbf{Procrustes alignment}: Find optimal rotation to match successive
    eigenvector matrices
  \item \textbf{Subspace distance}: $d(P, P') = \|P - P'\|_F$ is continuous and
    policy-independent
\end{itemize}

\subsubsection{Fail-Closed Demo}

\begin{figure}[t]
  \centering
  \includegraphics[width=0.7\textwidth]{figures/spectral/fail_closed.png}
  \caption{Fail-closed eigenvector export. When the eigenvalue gap is below
    threshold, the system exports only the cluster projector (debt), not
    individual eigenvectors. This prevents downstream consumers from using
    unreliable basis vectors.}
  \label{fig:fail-closed}
\end{figure}

\Cref{fig:fail-closed} shows the fail-closed policy in action. When eigenvalues
are within tolerance $\theta$, the system:
\begin{itemize}
  \item Refuses to export individual eigenvectors
  \item Exports the cluster projector with explicit debt metadata
  \item Signals to downstream consumers that basis ambiguity exists
\end{itemize}

\subsection{Subspace-First Analysis}

The experiments suggest a ``subspace-first'' approach to spectral analysis:

\begin{enumerate}
  \item Compute eigendecomposition
  \item Cluster eigenvalues by gap (e.g., $|\lambda_i - \lambda_j| < \theta$)
  \item Export cluster projectors, not individual eigenvectors
  \item If individual vectors are needed, require explicit policy choice
    with documented debt
\end{enumerate}

This approach preserves the meaningful information (subspace structure) while
making explicit the arbitrary choices (basis within subspace).

\subsection{Applications}

\paragraph{Principal Component Analysis.} Export the projector onto the
top-$k$ eigenspace rather than individual principal components. Comparison
across datasets uses subspace angles (e.g., Grassmann distance) rather than
vector correlations.

\paragraph{Spectral Clustering.} The cluster structure depends on the eigenspace,
not the specific eigenvector basis. Using projectors makes clustering results
reproducible across implementations.

\paragraph{Dynamical Systems.} In stability analysis, the stable/unstable/center
subspaces are the meaningful objects. Tracking these subspaces over parameter
changes avoids spurious discontinuities from eigenvector flips.
