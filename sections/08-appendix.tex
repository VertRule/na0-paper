\section{Proof Details}
\label{sec:appendix-proofs}

\subsection{Proof of Proposition~\ref{prop:pts-debt}}

We show that the remainder-channel PTS reduces to the evolved debt's projection
onto the remainder channel, using the typed formulation from \Cref{def:pts_typed}.

\begin{proof}
Let $\Pi$ be a policy with remainder extractor $r_\Pi: \mathcal{X} \to \mathcal{R}_\Pi$,
lift $\ell_\Pi: \mathcal{R}_\Pi \to \mathcal{X}$, and induced debt map
$q_\Pi(X) = X - \ell_\Pi(r_\Pi(X))$.

Assume additive reconstruction: $X = q_\Pi(X) + \ell_\Pi(r_\Pi(X))$.
Assume $U: \mathcal{X} \to \mathcal{X}$ is linear.

Write $R = r_\Pi(X) \in \mathcal{R}_\Pi$ and $D = q_\Pi(X) \in \mathcal{X}$.
By additive reconstruction: $X = D + \ell_\Pi(R)$.

Applying linearity of $U$:
\begin{equation}
  U(X) = U(D + \ell_\Pi(R)) = U(D) + U(\ell_\Pi(R)).
\end{equation}

The remainder-channel PTS (\Cref{def:pts_typed}) is:
\begin{align}
  \mathrm{PTS}_R(\Pi, U, X)
    &= r_\Pi(U(X)) - r_\Pi(U(\ell_\Pi(R))) \\
    &= r_\Pi(U(D) + U(\ell_\Pi(R))) - r_\Pi(U(\ell_\Pi(R))).
\end{align}

In the additive model, $r_\Pi$ is linear on $\mathcal{X}$ (since it extracts the
remainder component of an additive decomposition). Therefore:
\begin{align}
  \mathrm{PTS}_R(\Pi, U, X)
    &= r_\Pi(U(D)) + r_\Pi(U(\ell_\Pi(R))) - r_\Pi(U(\ell_\Pi(R))) \\
    &= r_\Pi(U(D)) \\
    &= r_\Pi(U(q_\Pi(X))).
\end{align}
This completes the proof.
\end{proof}

\subsection{Nakajima-Zwanzig Derivation Sketch}

For completeness, we sketch the Nakajima-Zwanzig derivation in $\NA$ language.

Let $\rho(t)$ be the full system-environment density matrix. Define projection:
\begin{equation}
  \proj \rho = \rho_S \otimes \rho_E^{\text{eq}}
\end{equation}
where $\rho_S = \mathrm{Tr}_E[\rho]$ and $\rho_E^{\text{eq}}$ is a reference
environment state.

The complement is $\mathcal{Q} = 1 - \proj$, projecting onto the ``irrelevant''
(correlated) part.

The Liouville-von Neumann equation:
\begin{equation}
  \frac{d\rho}{dt} = -i\mathcal{L}\rho = -i(\mathcal{L}_S + \mathcal{L}_E + \mathcal{L}_I)\rho
\end{equation}

Applying $\proj$ and $\mathcal{Q}$ separately and solving for the irrelevant part:
\begin{equation}
  \mathcal{Q}\rho(t) = e^{-i\mathcal{Q}\mathcal{L}t}\mathcal{Q}\rho(0)
  - i\int_0^t e^{-i\mathcal{Q}\mathcal{L}(t-\tau)} \mathcal{Q}\mathcal{L}\proj\rho(\tau)\, d\tau
\end{equation}

Substituting back:
\begin{equation}
  \frac{d\proj\rho}{dt} = -i\proj\mathcal{L}\proj\rho
  - \int_0^t K(t-\tau)\proj\rho(\tau)\, d\tau + I(t)
\end{equation}

where:
\begin{align}
  K(t) &= \proj\mathcal{L}\mathcal{Q} e^{-i\mathcal{Q}\mathcal{L}t} \mathcal{Q}\mathcal{L}\proj
  \quad \text{(memory kernel)} \\
  I(t) &= -i\proj\mathcal{L}\mathcal{Q} e^{-i\mathcal{Q}\mathcal{L}t} \mathcal{Q}\rho(0)
  \quad \text{(inhomogeneity)}
\end{align}

In $\NA$ terms:
\begin{itemize}
  \item The first term $\proj\mathcal{L}\proj\rho$ is the ``naive'' projected dynamics
  \item $K(t)$ encodes back-action from debt ($\mathcal{Q}\rho$)
  \item $I(t)$ encodes initial debt (correlations at $t=0$)
\end{itemize}

\section{Experimental Details}
\label{sec:appendix-experimental}

\subsection{Signal Processing Experiment}

\paragraph{Data Generation.}
Synthetic signals are generated with:
\begin{itemize}
  \item Domain: $x \in [0, 1]$, 500 points
  \item Background: $B = 1.0$
  \item Foreground: Gaussian envelope (width $0.3$) with 3--6 sub-blobs,
    amplitude 3--8 (varied per trial)
  \item Instrument drift: Quadratic polynomial, random coefficients $\sim U(-0.5, 0.5)$
  \item Noise: Gaussian, $\sigma = 0.05$
  \item Edge mask: 15\% on each side
\end{itemize}

\paragraph{Baseline Fitting.}
Polynomial fit to edge regions only, degrees 1, 2, 3, 6. Least-squares
fit with optional Tikhonov regularization (penalty $\lambda$ on coefficient
magnitudes, excluding constant term).

\paragraph{Background Estimation.}
After baseline subtraction, $\hat{B}$ is estimated as the mean of the
residual in the central (non-edge) region.

\paragraph{Reproducibility.}
All random seeds are fixed. Running the script with identical parameters
produces identical outputs (verified via SHA-256 hash of output arrays).

\subsection{Spectral Experiment}

\paragraph{Sign-Flip Test.}
\begin{itemize}
  \item Generate random SPD matrix $A = X X^T$ where $X \sim N(0, 1)$
  \item Compute eigendecomposition with different LAPACK driver seeds
  \item Compare eigenvectors (expect sign flips) and projectors (expect identity)
\end{itemize}

\paragraph{Degeneracy Test.}
\begin{itemize}
  \item Construct $A$ with eigenvalues $(1, 1+\epsilon, 2)$, $\epsilon = 10^{-4}$
  \item Perturb by $\delta = 10^{-6}$ symmetric noise
  \item Measure eigenvector angle change vs projector angle change
\end{itemize}

\paragraph{Procrustes Tracking.}
\begin{itemize}
  \item Generate time series of covariance matrices via random walk on SPD manifold
  \item Track eigenvectors with raw export vs Procrustes alignment
  \item Compute subspace distance between consecutive frames
\end{itemize}

\subsection{Jaynes-Cummings Experiment}

\paragraph{Parameters.}
\begin{itemize}
  \item Coupling: $g = 1.0$ (units where $\hbar = 1$)
  \item Initial photon number: $n \in \{0, 1, 2, 3\}$
  \item Time range: $[0, 2\pi]$, 33 points
  \item Output precision: 6 decimal places
\end{itemize}

\paragraph{Observables.}
\begin{itemize}
  \item $P_e^{\text{full}} = \cos^2(\Omega_n t)$
  \item $P_e^{\text{naive}} = 1$
  \item $P_e^{\text{corrected}} = 1 + \int_0^t (-\Omega_n \sin(2\Omega_n \tau))\, d\tau$
  \item Purity: $\mathrm{Tr}[\rho_A^2] = \cos^4(\Omega_n t) + \sin^4(\Omega_n t)$
  \item Entropy: $S(\rho_A) = -\lambda_+ \log_2 \lambda_+ - \lambda_- \log_2 \lambda_-$
    where $\lambda_\pm = (1 \pm \sqrt{2\mathrm{Tr}[\rho_A^2]-1})/2$
\end{itemize}

\section{Determinism Contract}
\label{sec:appendix-determinism}

For NA0 to serve as a reproducibility foundation, debt must be deterministically
serializable. We specify the following contract:

\begin{definition}[NA0 Determinism Contract]
  An NA0 implementation satisfies the \emph{determinism contract} if:
  \begin{enumerate}
    \item \textbf{Canonical encoding}: Debt $D$ and remainder $R$ admit a canonical
      byte-level encoding (e.g., IEEE 754 floats in little-endian, sorted keys for
      dictionaries, lexicographic ordering for sets).
    \item \textbf{Hash stability}: $\mathrm{SHA256}(\mathrm{encode}(D, R, \Pi))$ is
      identical across runs with identical inputs, regardless of execution order
      or parallelism.
    \item \textbf{Reconstruction idempotence}: $\mathrm{Rec}_\Pi(\mathrm{decode}(\mathrm{encode}(\NA\langle D, R; \Pi\rangle))) = \mathrm{Rec}_\Pi(\NA\langle D, R; \Pi\rangle)$.
    \item \textbf{No hidden state}: The policy $\Pi$ must declare all parameters
      affecting the projection (thresholds, hyperparameters, random seeds if any).
  \end{enumerate}
\end{definition}

\paragraph{Implementation notes.}
\begin{itemize}
  \item Use fixed-precision float formatting (e.g., 6 decimal places) in human-readable exports
  \item Use sorted dictionary keys in JSON/CSV exports
  \item Pin random seeds and document them as policy parameters
  \item Avoid unordered iteration (e.g., Python's pre-3.7 \texttt{dict}, \texttt{set})
\end{itemize}

\paragraph{Benchmark harness.}
The accompanying \texttt{benchmark/} directory provides a harness that:
\begin{itemize}
  \item Runs spectral, signal, and quantum benchmarks
  \item Outputs pass/fail status with metrics and thresholds
  \item Computes SHA-256 hashes of outputs for verification
  \item Verifies determinism by comparing hashes across runs
\end{itemize}
Execute with \texttt{make bench} or \texttt{python na0\_benchmark.py}.

\section{Notation Reference}
\label{sec:appendix-notation}

\begin{center}
\begin{tabular}{cl}
\toprule
Symbol & Meaning \\
\midrule
$\NA\langle \debt, \remainder; \policy \rangle$ & NA0 object with debt, remainder, policy \\
$\debt$ & Discarded information (debt) \\
$\remainder$ & Retained value (remainder) \\
$\policy$ & Projection policy \\
$\proj_\policy$ & Projection operator under policy $\policy$ \\
$\mathrm{Total}(\cdot)$ & Totalization: extract remainder, discard debt \\
$\mathrm{PTS}(\proj, U, X)$ & Projection timing sensitivity \\
$\remainder^{\langle\debt;\policy\rangle}$ & Debt-attached remainder notation \\
$\zeta(s)$ & Riemann zeta function \\
$\rho_S$ & Reduced density matrix of system \\
$K(t)$ & Nakajima-Zwanzig memory kernel \\
$P_i = v_i v_i^T$ & Eigenvector projector \\
$d(P, P')$ & Subspace distance (Frobenius norm) \\
\bottomrule
\end{tabular}
\end{center}
